\documentclass[a4paper,8pt]{report}

\usepackage[francais]{babel}
\usepackage[T1]{fontenc}
\usepackage[latin1]{inputenc}

\usepackage[left=2cm]{geometry}
\usepackage{amsmath,amssymb,mathrsfs}
\usepackage{graphicx}
\usepackage[table,xcdraw]{xcolor}

%Utilisation des codes sources en C
\usepackage{listings}
\lstset{
  language=C,
  language=Bash,
  basicstyle=\footnotesize,
  numbers=left,
  numberstyle=\normalsize,
  numbersep=7pt,
}

%titre pdf
\title{{\LARGE Algorithmique Avanc\'ee \\ Devoir de Programmation : Tries}}
\author{Bielle Benjamin - 2900825}

\graphicspath{{graph/}} 

\begin{document}

\maketitle
\renewcommand{\contentsname}{Sommaire}
\tableofcontents

%%%%%%%%%%%%%%%%%%%%%%%%%%%%%%%%%%%%%%%%%%%%%%%%%%%%%%%%%%%%%%%%%%%%%%%%%%%%%%%%%%%%%%%%%%%%%%%%%%%%%%%%%%%%%
% STRUCTURES
%%%%%%%%%%%%%%%%%%%%%%%%%%%%%%%%%%%%%%%%%%%%%%%%%%%%%%%%%%%%%%%%%%%%%%%%%%%%%%%%%%%%%%%%%%%%%%%%%%%%%%%%%%%%%
\chapter{Structures}
\section*{Arbres de la Briandais}\label{sec:name}
\addcontentsline{toc}{section}{Arbres de la Briandais}

Parmi le code ASCII nous pouvons d\'eterminer plusieurs marqeurs de fin de chain (ou de fin d'un mot).\\
Par exemple, les carat\`eres '*' ou '\#' mais ici nous prendrons le caract\`ere '\textbackslash0' le marqueur de fin de chaine ainsi la programmation de notre projet en sera facilit\'ee.

\medskip
Pour nous aider dans la construction d'un arbre de la Briandais, nous allons utiliser des primitives de base :\\
\begin{itemize}
\item newBRDtree (cl\'e : Mot, fils : BRDtree, fr\`ere : BRDtree).\\
  $\rightarrow$ retourne un arbre de la Briandais (BRDtree).\\
\item buildBRDtree (m : Mot).\\
  $\rightarrow$ retourne un arbre de la Briandais construit \`a partir du mot m (BRDtree).\\
\item addBRDtree (m : Mot, Tree : BRDtree).\\
  $\rightarrow$ ajoute un mot \`a un arbre de la Briandais (BRDtree).\\
\item emptyBRDtree ().\\
  $\rightarrow$ retourne un arbre de la Briandais vide.\\
\item isEmpty (arbre : BRDtree).\\
  $\rightarrow$ test si l'arbre est vide.\\
\item head (m : Mot).\\
  $\rightarrow$ retourne la premi\`ere lettre du mot m.\\
\item tail (m : Mot).\\
  $\rightarrow$ idem \`a la fonction head, retourne le reste des lettres du mot m (tout sauf la premi\`re lettre).\\
\end{itemize}

Nous allons construire un arbre de la Briandais \`a partir du texte \textbf{exemple de base}:\\
\textit{"A quel genial professeur de dactylographie sommes nous redevables de la superbe phrase ci dessous, un modele du genre, que toute dactylo connait par coeur puisque elle fait appel a chacune des touches du clavier de la machine a ecrire ?"}

\begin{figure}[H]
  \centering
  \includegraphics[width=0.8\textwidth]{briandais.png}
  \caption{Arbre de la Briandais}
  \label{fig:Arbre de la Briandais}
\end{figure}
\medskip
Pour avoir une image plus graphique de cet arbre, veuillez voir la figure \ref{fig:Arbre de la Briandais} page~\pageref{fig:Arbre de la Briandais}

\section*{Tries Hybrides}\label{sec:name}
\addcontentsline{toc}{section}{Tries Hybrides}

Comme pour l'arbre de la Briandais, nous allons d\'efinir des primitives nous permettant de constuire notre trie hybride :\\
\begin{itemize}
\item Hybrid (cl\'e : Mot, valeur : Mot, inf\'erieur : THybrid, sup\'erieur : THybrid, \'egale : THybrid).\\
  $\rightarrow$ retourne un trie hybride (THybrid).\\
\item emptyTHybrid ().\\
  $\rightarrow$ retourne un trie hybride vide.\\
\item isEmpty (trie : THybrid).\\
  $\rightarrow$ test si le trie est vide.\\
\item buildTHybrid (m : Mot).\\
  $\rightarrow$ retourne le trie hybride du mot m.\\
\item addTHybrid (m : Mot, t : THybrid).\\
  $\rightarrow$ ajoute le mot m au trie hybride t.\\
\end{itemize}

Idem \`a la premi\`ere partie (arbre de la Briandais), nous allons construire un trie hybride \`a partir du texte \textbf{exemple de base}:\\
\begin{figure}[H]
  \centering
  \includegraphics[width=0.8\textwidth]{hybrid.png}
  \caption{Trie Hybride}
  \label{fig:Trie Hybride}
\end{figure}
\medskip
Pour avoir une image plus graphique de ce trie, veuillez voir la figure \ref{fig:Trie Hybride} page~\pageref{fig:Trie Hybride}

%%%%%%%%%%%%%%%%%%%%%%%%%%%%%%%%%%%%%%%%%%%%%%%%%%%%%%%%%%%%%%%%%%%%%%%%%%%%%%%%%%%%%%%%%%%%%%%%%%%%%%%%%%%%%
% FONCTIONS AVANCEES
%%%%%%%%%%%%%%%%%%%%%%%%%%%%%%%%%%%%%%%%%%%%%%%%%%%%%%%%%%%%%%%%%%%%%%%%%%%%%%%%%%%%%%%%%%%%%%%%%%%%%%%%%%%%%
\chapter{Fonctions avanc\'ees pour chacune des structures}

\section*{Fonctions pour un arbre de la Briandais}\label{sec:name}
\addcontentsline{toc}{section}{Fonctions pour un arbre de la Briandais}

fonction de recherche d'un mot dans un dictionnaire : \textit{Recherche(arbre,mot) $\rightarrow$ bool\'een}.
\begin{verbatim}
  int searchBRDtree(char *word, BRDtree *Tree)
  {
    if(strcmp(word, "") == 0)
    {
      return isEmpty(T);
    }
    else if(isEmpty(T)) return FALSE;
    if(head(word) < T->key) return FALSE;
    else if(head(word) == T->key) return searchBRDtree(tail(word), T->child);
    else return searchBRDtree(word, T->next);
  }
\end{verbatim}

fonction qui compte les mots pr\'esents dans le dictionnaire : \textit{ComptageMots(arbre) $\rightarrow$ entier}.
\begin{verbatim}
  int countWordsBRDtree(BRDtree *Tree)
  {
    if(Tree == NULL)
       return 0;
    if(Tree->key == '\0')
       return 1 + countWordsBRDtree(Tree->next);

    return countWordsBRDtree(Tree->child) + countWordsBRDtree(Tree->next);
  }
\end{verbatim}

fonction qui liste les mots du dictionnaire dans l'ordre alphab\'etique : \textit{ListeMots(arbre) $\rightarrow$ liste[mots]}.
\begin{verbatim}
  void addWordList(char *word, wordList **list)
  {
     wordList *new = (wordList *) malloc(sizeof(wordList));
     new->word = word;
     new->next = *list;
     *list = new;
  }

  void insideListWordsBRDtree(BRDtree *Tree, wordList **list, char *word)
  {
     if(Tree == NULL)
        return;
     int n = strlen(word);
     char *newWord = (char *) malloc(sizeof(char) * (n + 1));
     int i;
     for(i = 0; i < n; i++)
         newWord[i] = word[i];
     newWord[i] = Tree->key;
     if(Tree->key == '\0')
         addWordList(newWord, list);
     insideListWordsBRDtree(Tree->child, list, newWord);
     insideListWordsBRDtree(Tree->next, list, word);
  }

  wordList *listWordsBRDtree(BRDtree *Tree)
  {
     wordList *list = (wordList *) malloc(sizeof(wordList));
     list->word = "";        
     list->next = NULL;
     insideListWordsBRDtree(Tree, &list, "");
     
     return list;
  }
\end{verbatim}

fonction qui compte les pointeurs vers NIL : \textit{ComptageNil(arbre) $\rightarrow$ entier}.
\begin{verbatim}
  int countNullBRDtree(BRDtree *Tree)
  {
    if(Tree == NULL)
       return 1;
    if(isEmpty(Tree))
       return 2;
    return countNullBRDtree(Tree->child) + countNullBRDtree(Tree->next);
  }
\end{verbatim}

fonction qui calcule la hauteur de l'arbre : \textit{Hauteur(arbre) $\rightarrow$ entier}.
\begin{verbatim}
  int heightBRDtree(BRDtree *Tree)
  {
     if(isEmpty(Tree)) return 0;
     return max(1 + heightBRDtree(Tree->child), heightBRDtree(Tree->next));
  }
\end{verbatim}

fonction qui calcule la profondeur moyenne des feuilles de l'arbre : \textit{ProfondeurMoyenne(arbre) $\rightarrow$ entier}.
\begin{verbatim}
  void insideAverage(BRDtree *Tree, int *level, int *count, int currLevel)
  {
    if(Tree == NULL) return;
    (*level) += currLevel;
    (*count)++;
    if(Tree->child != NULL)
       insideAverage(Tree->child, level, count, currLevel + 1);
    if(Tree->next != NULL)
       insideAverage(Tree->next, level, count, currLevel);
  }

  double averageLevelBRDtree(BRDtree *Tree)
  {
    int  level = 0, count = 0, currLevel = 1;
    insideAverage(Tree, &level, &count, currLevel);
    if(count == 0) return 0.0;
    return (double) level / count;
  }
\end{verbatim}

fonction qui prend un mot A en argument et qui indique de combien de mots du dictionnaire le mot A est le pr\'efixe : \textit{Prefixe(arbre,mot) $\rightarrow$ entier}.
\begin{verbatim}
  int countPrefixBRDtree(char *prefix, BRDtree *Tree)
  {
     if(strcmp(prefix, "") == 0)
        return countWordsBRDtree(Tree);
     if(head(prefix) < Tree->key)
        return 0;
     else if (head(prefix) > Tree->key)
        return countPrefixBRDtree(prefix, Tree->next);
     else
        return countPrefixBRDtree(tail(prefix), Tree->child);
  }
\end{verbatim}

fonction qui prend un mot en argument et qui le supprime de l'arbre \textbf{s'il y figure} : \textit{Suppression(arbre,mot) $\rightarrow$ arbre}.
\begin{verbatim}
  BRDtree *delBRDtree(char *word, BRDtree *Tree)
  {
     if(Tree == NULL)
        return NULL;
     if(isEmpty(Tree) && strcmp(word, "") == 0)
        return Tree->next;
     if(Tree->key == head(word))
     {
        BRDtree *child = delBRDtree(tail(word), Tree->child);
        if(child == NULL)
            return Tree->next;
        else
            return BRD(Tree->key, child, Tree->next);
     }
     else if(Tree->key > head(word))
        return Tree;
     else
        return BRD(Tree->key, Tree->child, delBRDtree(word, Tree->next));
  }
\end{verbatim}

%%%%%%%%%%%%%%%%%%%%%%%%%%%%%%%%%%%%%%%%%%%%%%%%%%%%%%%%%%%%%%%%%%%%%%%%%%%%%%%%%%%%%%%%%%%%%%%%%%%%%%%%%%%%%
\section*{Fonctions pour un trie hybride}\label{sec:name}
\addcontentsline{toc}{section}{Fonctions pour un trie hybride}

fonction de recherche d'un mot dans un dictionnaire : \textit{Recherche(arbre,mot) $\rightarrow$ bool\'een}.
\begin{verbatim}
int searchTHybrid(char *word, THybrid *Trie)
{
  #ifdef VERBOSE
  printf("WD [%c]\n",*word);
  #endif
  if (Trie == NULL) return FALSE;
  else if (strcmp(word,"") == 0)
    {
      if (Trie->val == NOTEMPTY) return TRUE;
      else                       return FALSE;
    }
  else
    {
      if (*word < Trie->key)      return searchTHybrid(word, Trie->inf);
      else if (*word > Trie->key) return searchTHybrid(word, Trie->sup);
      else                        return searchTHybrid(++word, Trie->eq);
    }
}
\end{verbatim}

fonction qui compte les mots pr\'esents dans le dictionnaire : \textit{ComptageMots(arbre) $\rightarrow$ entier}.
\begin{verbatim}
int countWordsTHybrid(THybrid *Trie)
{
  if(Trie == NULL)
    return 0;
  else 
    {
      if(Trie->val == EMPTY)
	return countWordsTHybrid(Trie->eq) 
	  + countWordsTHybrid(Trie->inf) 
	  + countWordsTHybrid(Trie->sup);
      else
	return 1 + countWordsTHybrid(Trie->eq) 
	  + countWordsTHybrid(Trie->inf) 
	  + countWordsTHybrid(Trie->sup);
    }
}
\end{verbatim}

fonction qui liste les mots du dictionnaire dans l'ordre alphab\'etique : \textit{ListeMots(arbre) $\rightarrow$ liste[mots]}.
\begin{verbatim}
void insideListWordsTHybrid(THybrid *Trie, wordList **list, char *word)
{
  if (Trie == NULL) return;
  int size=strlen(word),i;
  char *nw = (char *)malloc(sizeof(char)*(size+1));

  for (i=0; i<size; i++)
    nw[i] = word[i];
  nw[i] = Trie->key;

  if (Trie->val == NOTEMPTY)
    addWordList(nw, list);

  insideListWordsTHybrid(Trie->eq, list, nw);
  insideListWordsTHybrid(Trie->inf, list, nw);
  insideListWordsTHybrid(Trie->sup, list, nw);
}

wordList *listWordsTHybrid(THybrid *Trie)
{
  wordList *list = (wordList *) malloc(sizeof(wordList));
  list->next = NULL;
  insideListWordsTHybrid(Trie, &list, "");
  return list;
}
\end{verbatim}

fonction qui compte les pointeurs vers NIL : \textit{ComptageNil(arbre) $\rightarrow$ entier}.
\begin{verbatim}
int countNullTHybrid(THybrid *Trie)
{
  if (Trie == NULL) return 1;
  else
    return  countNullTHybrid(Trie->eq)
      + countNullTHybrid(Trie->inf)
      + countNullTHybrid(Trie->sup);
}
\end{verbatim}

fonction qui calcule la hauteur de l'arbre : \textit{Hauteur(arbre) $\rightarrow$ entier}.
\begin{verbatim}
int heightTHybrid(THybrid *Trie)
{
  if (Trie == NULL) return 0;
  else
    return 1 + maxOfThree(heightTHybrid(Trie->eq), heightTHybrid(Trie->inf), heightTHybrid(Trie->sup));
}
\end{verbatim}

fonction qui calcule la profondeur moyenne des feuilles de l'arbre : \textit{ProfondeurMoyenne(arbre) $\rightarrow$ entier}.
\begin{verbatim}
void insideAverageTHybrid(THybrid *Trie, int *level, int *count, int currLevel)
{
  if(Trie == NULL) return;
  (*level) += currLevel;
  (*count)++;
  if(Trie->inf != NULL)
    insideAverageTHybrid(Trie->inf, level, count, currLevel + 1);
  if(Trie->eq != NULL)
    insideAverageTHybrid(Trie->eq, level, count, currLevel + 1);
  if(Trie->sup != NULL)
    insideAverageTHybrid(Trie->sup, level, count, currLevel + 1);
}

double averageLevelTHybrid(THybrid *Trie)
{
  int level=0, count=0, currLevel=0;
  insideAverageTHybrid(Trie, &level, &count, currLevel);
  if (count == 0) return (double) 0;
  return (double) level/count;
}
\end{verbatim}

fonction qui prend un mot A en argument et qui indique de combien de mots du dictionnaire le mot A est le pr\'efixe : \textit{Prefixe(arbre,mot) $\rightarrow$ entier}.
\begin{verbatim}
int countPrefixTHybrid(char *prefix, THybrid *Trie)
{
  if (strcmp(prefix,"") == 0)
    {
      return countWordsTHybrid(Trie);
    }
  if (*prefix < Trie->key)
    {
      return countPrefixTHybrid(prefix, Trie->inf);
    }
  else if (*prefix > Trie->key)
    {
      return countPrefixTHybrid(prefix, Trie->sup);
    }
  else
    {
      return countPrefixTHybrid(++prefix, Trie->eq);
    }
}
\end{verbatim}

fonction qui prend un mot en argument et qui le supprime de l'arbre \textbf{s'il y figure} : \textit{Suppression(arbre,mot) $\rightarrow$ arbre}.
\begin{verbatim}
THybrid *delTHybrid(char *word, THybrid *Trie)
{
  if(Trie == NULL)
    return NULL;
  else if(strcmp(word,"") == 0)
    {
      if(Trie->val == NOTEMPTY)
	{
	  if(Trie->eq != NULL)
	    {
	      Trie->val = EMPTY;
	    }
	  else if(Trie->inf != NULL)
	    {
	      Trie->val = EMPTY;
	      Trie->eq = Trie->inf;
	      Trie->inf = NULL;
	    }
	  else if(Trie->sup != NULL)
	    {
	      Trie->val = EMPTY;
	      Trie->eq = Trie->sup;
	      Trie->sup = NULL;
	    }
	  else
	    {
	      free(Trie);
	      Trie = NULL;
	    }
	}
      return Trie;
    }
  else
    {
      if(*word < Trie->key)
	{
	  Trie->inf = delTHybrid(word, Trie->inf);
	}
      else if(*word > Trie->key)
	{
	  Trie->sup = delTHybrid(word, Trie->sup);
	}
      else
	{
	  Trie->eq = delTHybrid(++word, Trie->eq);
	}
      if(Trie->inf == NULL && Trie->sup == NULL && Trie->eq == NULL)
	{
	  free(Trie);
	  Trie = NULL;
	}
      else if(Trie->eq == NULL && Trie->inf != NULL)
	{
	  Trie->eq = Trie->inf;
	  Trie->inf = NULL;
	}
      else if(Trie->eq == NULL && Trie->sup != NULL)
	{
	  Trie->eq = Trie->sup;
	  Trie->sup = NULL;
	}
      return Trie;
    }
}
\end{verbatim}

%%%%%%%%%%%%%%%%%%%%%%%%%%%%%%%%%%%%%%%%%%%%%%%%%%%%%%%%%%%%%%%%%%%%%%%%%%%%%%%%%%%%%%%%%%%%%%%%%%%%%%%%%%%%%
% FONCTIONS COMPLEXES
%%%%%%%%%%%%%%%%%%%%%%%%%%%%%%%%%%%%%%%%%%%%%%%%%%%%%%%%%%%%%%%%%%%%%%%%%%%%%%%%%%%%%%%%%%%%%%%%%%%%%%%%%%%%%
\chapter{Fonctions complexes}

fonction qui prend deux arbres de la Briandais en argument et les fusionnent en un troisi\`eme : \textit{mergeBRDtree(arbre1,arbre2) $\rightarrow$ BRDtree}.
\begin{verbatim}
BRDtree *mergeBRDtree(BRDtree *Tree1, BRDtree *Tree2)
{
  if(Tree1 == NULL) return Tree2;
  if(Tree2 == NULL) return Tree1;
  
  if(Tree2->key < Tree1->key)
    return newBRDtree(Tree2->key, Tree2->child, mergeBRDtree(Tree1, Tree2->next));
  else if(Tree2->key > Tree1->key)
    return newBRDtree(Tree1->key, Tree1->child, mergeBRDtree(Tree1->next, Tree2));
  else
    return newBRDtree(Tree1->key, mergeBRDtree(Tree1->child, Tree2->child),
		                  mergeBRDtree(Tree1->next, Tree2->next));
}
\end{verbatim}

fonction permettant de passer d'un trie hybride \`a un arbre de la Briandais : \textit{THybridToBRDtree (trie) $\rightarrow$ BRDtree}.
\begin{verbatim}
void tmpToBRDtree (THybrid *trie, BRDtree *tree, char *word)
{
  char *nw;
  int size,i;
  
  if (trie == NULL){return;}
 
  size = strlen(word);
  nw = (char *)malloc(sizeof(char)*(size+1));
  
  for (i=0; i<size; i++)
    nw[i] = word[i];
  nw[i] = trie->key;
  
  if (trie->val == NOTEMPTY)
    tree = addBRDtree(nw,tree);
  
  tmpToBRDtree(trie->eq, tree, nw);
  tmpToBRDtree(trie->inf, tree, word);
  tmpToBRDtree(trie->sup, tree, word);
}

BRDtree *THybridToBRDtree (THybrid *Trie)
{
  BRDtree *brd = emptyBRDtree();
  tmpToBRDtree(Trie, brd, "");
  return brd;
}
\end{verbatim}

fonction permettant de passer d'un arbre de la Briandais \`a un trie hybride: \textit{BRDtreeToTHybrid (arbre) $\rightarrow$ THybrid}.
\begin{verbatim}
\end{verbatim}

fonction permettant de r\'equilibrer un trie hybride: \textit{rebalancing (trie) $\rightarrow$ THybrid}.
\begin{verbatim}
\end{verbatim}

%%%%%%%%%%%%%%%%%%%%%%%%%%%%%%%%%%%%%%%%%%%%%%%%%%%%%%%%%%%%%%%%%%%%%%%%%%%%%%%%%%%%%%%%%%%%%%%%%%%%%%%%%%%%%
% COMPLEXITES
%%%%%%%%%%%%%%%%%%%%%%%%%%%%%%%%%%%%%%%%%%%%%%%%%%%%%%%%%%%%%%%%%%%%%%%%%%%%%%%%%%%%%%%%%%%%%%%%%%%%%%%%%%%%%
\chapter{Complexit\'es}
\section*{Arbres de la Briandais}\label{sec:name}

fonction de recherche d'un mot dans un arbre de la Briandais, dans le pire cas on parcours L caract\`eres (L correspond \`a la longueur du mot) donc O(L).\\
fonction de comptage de mots, on parcours tous l'arbre soit L caract\`eres et pour chaque L on parcours sa hauteur soit ln N donc O(L * ln N).\\
fonction liste de mots, on parcours simplement l'arbre donc O(L * ln N).\\
fonction de comptage des NULL, idem que comptage de mots.\\
fonction de calcul de la hauteur, on parcours tous l'arbre donc environ O(log2 N).\\
fonction de calcul de la profondeur moyenne de l'arbre, on parcours tous l'arbre donc O(L * ln N).\\
fonction de comptage de prefix, cela est d\'etermin\'ee par la hauteur de l'arbre pour le prefix recherch\'e donc O(log2 N).\\
fonction de suppression d'un mot dans l'arbre, cette fonction d\'epend aussi de la hauteur de l'arbre donc O(log2 N).\\

\section*{Tries Hybrides}\label{sec:name}

fonction de recherche d'un mot dans un trie hybride, dans le pire cas on parcours L caract\`eres (L correspond \`a la longueur du mot) plus sa haurteur donc O(L + ln N).\\
fonction de comptage de mots, on parcours tous le trie soit L caract\`eres et pour chaque L on parcours sa hauteur soit ln N donc O(L * ln N).\\
fonction liste de mots, on parcours simplement l'arbre donc O(L * ln N).\\
fonction de comptage des NULL, idem que comptage de mots.\\
fonction de calcul de la hauteur, on parcours tous l'arbre donc environ O(log2 N).\\
fonction de calcul de la profondeur moyenne de l'arbre, on parcours tous l'arbre donc O(L * ln N).\\
fonction de comptage de prefix, cela est d\'etermin\'ee par la hauteur de l'arbre pour le prefix recherch\'e donc O(log2 N).\\
fonction de suppression d'un mot dans l'arbre, cette fonction d\'epend aussi de la hauteur de l'arbre donc O(log2 N).\\


The time complexity of the ternary search tree operations is similar to that of binary search tree. i.e. the insertion, deletion and search operations take time proportional to the height of the ternary search tree. The space is proportional to the length of the string to be stored.

\section*{R\'esum\'e}\label{sec:name}

\begin{table}[h]
\begin{tabular}{ccccccccc}
                                                           & \cellcolor[HTML]{C0C0C0}search & \cellcolor[HTML]{C0C0C0}count words & \cellcolor[HTML]{C0C0C0}list words & \cellcolor[HTML]{C0C0C0}count Nil & \cellcolor[HTML]{C0C0C0}height & \cellcolor[HTML]{C0C0C0}average depth & \cellcolor[HTML]{C0C0C0}prefix & \cellcolor[HTML]{C0C0C0}delete word \\
\cellcolor[HTML]{C0C0C0}{\color[HTML]{656565} Briandais}   & \cellcolor[HTML]{EFEFEF}L      &    L * ln N                                 & \cellcolor[HTML]{EFEFEF}L * ln N           & L * ln N                                  & \cellcolor[HTML]{EFEFEF} $\log_2 N$      &         L * ln N                              & \cellcolor[HTML]{EFEFEF}   $\log_2 N$    &      $\log_2 N$                               \\
\cellcolor[HTML]{C0C0C0}{\color[HTML]{656565} Hybrid Trie} & \cellcolor[HTML]{EFEFEF}L + ln N &      L * ln N                                & \cellcolor[HTML]{EFEFEF}     L * ln N       &     L * ln N                               & \cellcolor[HTML]{EFEFEF}    $\log_2 N$   &              L * ln N                         & \cellcolor[HTML]{EFEFEF}    $\log_2 N$   &     $\log_2 N$                                \\
                                                           &                                &                                      &                                    &                                   &      
\end{tabular}
\end{table}

%%%%%%%%%%%%%%%%%%%%%%%%%%%%%%%%%%%%%%%%%%%%%%%%%%%%%%%%%%%%%%%%%%%%%%%%%%%%%%%%%%%%%%%%%%%%%%%%%%%%%%%%%%%%%
% ANNEXES
%%%%%%%%%%%%%%%%%%%%%%%%%%%%%%%%%%%%%%%%%%%%%%%%%%%%%%%%%%%%%%%%%%%%%%%%%%%%%%%%%%%%%%%%%%%%%%%%%%%%%%%%%%%%%
\chapter{Annexe}
\section*{Structures de donn\'ees}\label{sec:name}
\addcontentsline{toc}{section}{Structures de donn\'ees}

Ici nous d\'etaillerons les diff\'erentes structures de donn\'ees utilis\'ees dans ce projet.

\subsection*{Arbres de la Briandais}\label{sec:name}
\addcontentsline{toc}{subsection}{Arbres de la Briandais}

La structure de donn\'ees \textit{BRDtree} est la structure d\'efinissant un arbre de la Briandais.
\begin{verbatim}
  typedef struct BRDtree
  {
     char key; 
     struct BRDtree *child; 
     struct BRDtree *next; 
  }BRDtree;
\end{verbatim}

La structure de donn\'ees \textit{wordList} est la structure d\'efinissant une liste de mots pour un arbre de la Briandais.
\begin{verbatim}
  typedef struct wordList
  {
     char *word;
     struct wordList *next;
  }wordList;
\end{verbatim}

\subsection*{Tries Hybrides}\label{sec:name}
\addcontentsline{toc}{subsection}{Tries Hybrides}

La structure de donn\'ees \textit{THybrid} est la structure d\'efinissant un trie hybride.
\begin{verbatim}
  typedef struct THybrid
  {
     char key;
     char val;
     struct THybrid *inf;
     struct THybrid *eq;
     struct THybrid *sup;
  }THybrid;
\end{verbatim}

\addcontentsline{toc}{section}{\listfigurename}
\listoffigures
\newpage
%\section*{Bibliographie}\label{sec:name}
%\addcontentsline{toc}{section}{Bibliographie}

\end{document}
